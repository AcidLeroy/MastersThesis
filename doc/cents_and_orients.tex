\chapter{\label{ap:centroids}Retrieving Centroids and Orientations from Blobs}
This is a snippet of code that illustrates how we retrieve orientations and
centroids from blobs of grouped motion vectors. This code is used to extract
additional features from each video frame pair as described in the methods
section. 
\begin{minted}{cpp}
void UpdateCentroidAndOrientation(const cv::Mat& thresholded_image,
                                  cv::Mat* orientations, cv::Mat* centroids) {
  cv::Mat labels, stats, current_centroids;
  cv::connectedComponentsWithStats(thzresholded_image, labels, stats,
                                   current_centroids);
  // Don't care about background centroid, hence the range.
  if (centroids->empty()) {
    current_centroids(cv::Range(1, current_centroids.rows),
                      cv::Range(0, current_centroids.cols)).copyTo(*centroids);
  } else {
    cv::vconcat(*centroids,
                current_centroids(cv::Range(1, current_centroids.rows),
                                  cv::Range(0, current_centroids.cols)),
                *centroids);
  }

  std::vector<std::vector<cv::Point>> contours;
  cv::findContours(thresholded_image, contours, cv::RETR_LIST,
                   cv::CHAIN_APPROX_NONE);
  for (size_t i = 0; i < contours.size(); ++i) {
    // Can only fit an ellipse with 5 points, skip others
    if (contours[i].size() >= 5) {
      cv::RotatedRect result = cv::fitEllipse(contours[i]);
      orientations->push_back(result.angle);
    }
  }
}
\end{minted}
