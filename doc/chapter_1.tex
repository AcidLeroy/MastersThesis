\chapter{Introduction}
The advancing out of school learning in mathematics and engineering (AOLME)
project contains thousands of hours of high quality video data of children
learning how to program the Raspberry Pi. These videos are valuable to educators
to help guide the construction of well formed curricula to improve education to
children, especially in under represented communities. In order to analyze these
videos, researches in the education department at UNM  manually annotate these
videos which is an extremely tedious and unsustainable method for such a large
dataset. Because of these inhibitory factors, most of these encoded videos are
left untouched and unanalyzed, potentially leaving thousands of hours of
valuable information about the learning process unexplored and  underutilized.
Clearly there is a need for a tool to aid researchers in properly exploring
these video datasets efficiently.

\section{\label{section:motivation}Motivation}

Current methods in video analysis systems that attempt to solve the problem  as
presented in AOLME are extremely application dependent and are inadequate
computationally and in scalability to sufficiently investigate video datasets at such
a large scale. As such, there is a propensity for a system that is accurate,
scalable and flexible in nature to handle a variety of problems currently faced
with automated video analysis.

Computationally, there is clearly a need for video analysis methods that can be
efficiently implemented in heterogenous compute hardware (such as GPUS and
CPUS), and have said hardware function in a distributed environment. Being able
to leverage heterogenous computer hardware greatly increases the efficiency and
speed of certain, heavily used, video processing algorithms such as 2D
convolutions. Furthermore, having this system exist in a distributed environment
will greatly speed up ephemeral operations and makes it possible to scale up
future versions of ViDA seamlessly. Thus this thesis is motivated
by addressing deficiencies in both current analysis techniques used in AOLME sans
automation and those in big data analysis in video processing.

\section{\label{section:thesis_statement}Thesis Statement}
The thesis of this research is that it is possible to scale, accurately
classify and process videos using ViDA. Through the use of a handful of algorithms,
such as Farneback and Lucas-Kanade optical flow and SVMs powered by a highly scalable
computing architecture, we show that large amounts of video data can be
accurately classified.

\section{\label{section:contributions}Contributions}
\section{\label{section:summary}Summary}
