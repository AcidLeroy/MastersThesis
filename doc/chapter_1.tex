\chapter{Introduction}
\section{\label{section:motivation}Motivation}
The AOLME project contains thousands of hours of high quality video data of
children learning how to program the Raspberry Pi. These videos are valuable to
educators to help guide the construction of well formed curricula to  improve
education to children, especially in underprivileged communities. Currently,
researchers in the education department at UNM have no choice but to manually
annotate theses videos by hand, which is an extremely tedious and unsustainable
method of analysis for such large datasets. Present methods in video analysis
systems that attempt to solve this problem are extremely application dependent
and are inadequate computationally and in scalability to sufficiently tackle
video analysis at such a large scale. As such, there is a propensity for a
system that is accurate, scalable and flexible in nature to handle a variety of
problems currently faced with automated video annotation.

\section{\label{section:thesis_statement}Thesis Statement}
\section{\label{section:contributions}Contributions}
\section{\label{section:summary}Summary}
