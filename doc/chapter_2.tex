\chapter{Activity Classification in Videos}
One of the main objectives of this research is to provide a reliable method
for activity classification in videos. For this thesis, two activities of
interest to the College of Education are studied: when students are writing and when they
typing. The idea being that if we can prove that classification using ViDA
works for simple activities, we can extend the system to handle
other activities just by changing the training data and training a new classifier
on that data, making the system extremely flexible. This chapter reviews the basic techniques that are used
for reducing the feature space of videos so that any machine learning algorithm,
such as an SVM, can be used to classify the activities. It explains the basic
implementation of the Farneback optical flow algorithm \cite{farneback2003two}, also known as dense
optical flow, how the optical flow features are further reduced, and finally the
methods used to classify the activity in the video.

\section{\label{section:optical_flow_methods}Optical Flow Methods}
In this thesis we attempt to classify two types of activities in video,  typing
and writing. Since both of these activities involve motion, i.e. a change of
apparent structure position from one video frame to the next, optical flow
algorithms  are a suitable tool for attempting to extract germane features from
the video.

Currently, there are several varieties of optical flow algorithms that have been
published. We use both Lucas-Kanade \cite{lucas1981iterative} and the Farneback
\cite{farneback2003two}  optical flow algorithms to attempt to extract
important motion features from the AOLME videos. Although as we see with our
experiments, that the Farneback algorithm is better suited for pulling out
motion features that our unique to the particular motion for which we attempt
to find in our videos. In either case, both algorithms attempt to solve
Equation \ref{eq:delta_image}

\begin{equation}
I(x,y,t) = I(x+dx, y+dy, t+dt)
\label{eq:delta_image}
\end{equation}

where $I$ is the the image, $x$ \& $y$ are the column row coordinates respectively,
and $t$ is the time between two adjacent image frames. Taking the Talylor
series expansion of Equation \ref{eq:delta_image} results in Equation \ref{eq:taylor_expansion}


\begin{equation}
f_x u + f_y v + f_t = 0
\label{eq:taylor_expansion}
\end{equation}

where

\begin{equation}
f_x = \frac{\partial f}{\partial x} \; ; \; f_y = \frac{\partial f}{\partial y} \\
u = \frac{dx}{dt} \; ; \; v = \frac{dy}{dt}
\label{eq:taylor_expansion_partial}
\end{equation}


\subsection{\label{subsection:lucas_kanade} Lucas-Kanade Method}
For this thesis, we leverage a common C++ library that has already implemented
a specialized version of the general Lucas-Kanade algorithm which uses pyramids
to solve the optical flow at different scales of motion
 \cite{bouguet2001pyramidal}. This algorithm is provided freely in a C++ computer
 vision library known as OpenCV \cite{itseez2015opencv}. Essentially, this algorithm is much like the original
 paper published


\section{\label{section:feature_extraction}Feature Extraction from Optical Flow}
\section{\label{section:classification}Classifying the Reduced Feature Space}
