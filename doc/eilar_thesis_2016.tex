% Example template for using the unmeethesis style
% This example is for a Master's candidate in Mathematics
% It contains examples of front matter and most sections that the
% typical graduate student would need to include
% By: N. Doren 02/10/00
%     Minor mods by N. Doren 08/26/11

% Use the following specification for BOTTOM page numbering:
\documentclass[botnum, fleqn]{unmeethesis}
% OR
% Use the following specification for TOP page numbering:
% \documentclass[fleqn]{unmeethesis}

\begin{document}

  \frontmatter

  % Uncomment the next command if you see weird paragraph spacing:
  % That is, if you see paragraphs float with lots of white space
  % in between them:

  % \setlength{\parskip}{0.30cm}


  \title{Distributed Video Analysis  for the  Advancing  \\
  Out of School Learning in Mathematics and Engineering}

  \author{Cody Wilson Eilar}

  \degreesubject{M.S., Computer Engineering}

  \degree{Master of Science \\ Computer Engineering}

  \documenttype{Thesis}

  \previousdegrees{B.S., University of New Mexico, 2010}

  \date{July, \thisyear}

  \maketitle

  %\makecopyright
  %Copyright page is no longer necessary D. Murrell


  \begin{acknowledgments}
    \vspace{1.1in}
    I would like to thank my advisor, Professor Marios Pattichis, for his continuous
    support in reviewing my source code and for helping me shape my ideas for this thesis.
    I would also like to thank
    Elmyra Grelle for helping me stay on track with all the necessary requirements
    to graduate.
  \end{acknowledgments}

  \maketitleabstract %(required even though there's no abstract title anymore)

  \begin{abstract}
    We develop a distributed processing system for analyzing classroom videos to assess student learning while participating in the advancing out of school learning in mathematics and 
engineering (AOLME) project. We demonstrate our system in detecting writing and typing activities based on the Video Distributed Analysis (VIDA) system. VIDA uses a master node to 
optimally distribute video segments to specific devices (e.g., CPU, GPU) within each heterogeneous processing node. The results are then collected by the master node. We 
demonstrate the scalability and speedup of VIDA for analyzing videos with large image formats over different numbers of processing nodes.
    \clearpage %(required for 1-page abstract)
  \end{abstract}

  \tableofcontents
  \listoffigures
  \listoftables

  \begin{glossary}{Longest  string}
    \item[$a_{lm}$]
    Taylor series coefficients, where $l,m = \{0..2\}$
    \item[$A_{\bf{p}}$]
    Complex-valued scalar denoting the amplitude and phase.
    \item[$A^T$]
    Transpose of some relativity matrix.
  \end{glossary}

  \mainmatter

  \chapter{Introduction}
  \section{\label{section:overview}Overview}
  The classic approach to proving a theorem is some really difficult
  mathematics.  For the theory of relativity, I asked grandpa Al exactly
  how he proved it.  He gave me a few hints, including some stuff about
  rest mass and big electro-motive force.  I think he is really smart.
  \section{Conclusions}
  I conclude that this is a really short thesis.

  \chapter{Future Work}
  I'm sure my future work will consist of lots of other famous stuff.

  \chapter*{Appendices}

  \addcontentsline{toc}{chapter}{Appendices}
  % Next lines duplicated from .toc file and used to create mini
  % "Appendix Table of Contents," if desired:
  \contentsline {chapter}{\numberline {A}Proving $E=MC^2$}{4}
  \contentsline {chapter}{\numberline {B}Derivation of $A = \pi r^2$}{5}
  % End mini table of contents

  \appendix
  \chapter{Proving $E=MC^2$}
  I refer the reader to many of grandpa's famous books on this subject.
  \chapter{Derivation of $A = \pi r^2$}
  A circle is really a square without corners.  QED.

  %\bibliographystyle{AMS}
  %\bibliography{bibfile_name}

\end{document}
