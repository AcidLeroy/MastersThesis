\chapter{\label{ap:svm_classification}R Code for SVM Classification}
This sample code is the same code that was used in the master node for
classification of the features. It uses the leave one out strategy for
training and testing the optimal solution. This is referenced from the
Methods chapter in this thesis. 
\begin{minted}{r}
  ClassifyFeatures <- function(VideoHists){
    NoOfSamples <- length(VideoHists$Classification)

    # Build a factor of the correct classification:
    All_cl <- unlist(VideoHists$Classification);

    # Store 1 for wrong classification and 0 for correct.
    knnResult <- rep(1, times=NoOfSamples);
    svmResult <- rep(1, times=NoOfSamples);

    # Remove classification
    no_use = c("Filename", "Classification")
    features = GetAllExcept(VideoHists, no_use)

    # Create a leave one out classification approach
    for(i in 1:NoOfSamples)
    {
      # Set up training and testing data:
      trainData = lapply(features, function(x) x[-i]) # Remove i.
      testData  = lapply(features, function(x) x[i]) # One left out.

      #Combine data
      trainData = t(CombineFeatures(trainData, names(trainData)))
      testData = t(CombineFeatures(testData, names(testData)))

      # Prepare the labels for the training set:
      #  Optimal: k=1
      knnResult[i] <- knn (trainData, testData, All_cl[-i], k=3); # 3

      #**** With tuning ****#
      tune.out=tune(svm, trainData, All_cl[-i], , kernel="linear", ranges=list(cost=c(0.0001, 0.001, 0.01, 0.1, 1, 5, 10, 100, 10000)) , scale=FALSE)
      svmResult[i] <- predict(tune.out$best.model, testData);



      #***** Without tuning *****#
  #    model <- svm(All_cl[-i] ~ ., data=trainData, scale=FALSE);
  #    svmResult[i] <- predict(model, testData);
      cat("SVM result =  ", round(svmResult), "\n");
      cat("KNN result =  ", knnResult, "\n")

    }
\end{minted}
